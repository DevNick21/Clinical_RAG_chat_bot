\chapter{Code Listings and Implementation Details}
\label{appendix:code}

This appendix contains key code snippets, implementation details, and technical documentation for the Clinical RAG Chatbot system.

\section{Core RAG Pipeline Implementation}

The following sections present the main components of the Clinical RAG system implementation:

\subsection{Document Processing and Chunking}

The core document processing pipeline transforms raw clinical data into searchable chunks suitable for vector retrieval:

\begin{minted}{python}
# RAG_chat_pipeline/utils/data_provider.py
class DataProvider:
    """Abstraction layer for data source selection (real vs synthetic)"""
    
    def __init__(self):
        self.using_synthetic = not self._has_real_data()
        if self.using_synthetic:
            self._ensure_synthetic_data()
    
    def _has_real_data(self) -> bool:
        """Check if real MIMIC-IV data is available"""
        mimic_path = BASE / "mimic_sample_1000"
        required_files = [
            "admissions.csv_sample1000.csv",
            "diagnoses_icd.csv_sample1000.csv",
            "labevents.csv_sample1000.csv"
        ]
        return all((mimic_path / f).exists() for f in required_files)
    
    def load_chunked_docs(self):
        """Load processed document chunks from appropriate source"""
        if self.using_synthetic:
            return self._load_synthetic_chunks()
        else:
            return self._load_real_chunks()
    
    def _create_document_chunks(self, raw_data):
        """Transform raw clinical data into semantic chunks"""
        chunks = []
        for admission in raw_data:
            # Create header chunk
            header_content = self._format_header(admission)
            chunks.append(Document(
                page_content=header_content,
                metadata={
                    "subject_id": admission["subject_id"],
                    "hadm_id": admission["hadm_id"],
                    "section": "header"
                }
            ))
            
            # Process each clinical section
            sections = ["diagnoses", "procedures", "labs", 
                       "prescriptions", "microbiology"]
            for section in sections:
                if section in admission and admission[section]:
                    content = self._format_section(admission[section], section)
                    chunks.append(Document(
                        page_content=content,
                        metadata={
                            "subject_id": admission["subject_id"],
                            "hadm_id": admission["hadm_id"],
                            "section": section
                        }
                    ))
        return chunks
\end{minted}

\subsection{Vector Embeddings and Storage}

The embedding management system handles multiple embedding models and FAISS vector store creation:

\begin{minted}{python}
# RAG_chat_pipeline/core/embeddings_manager.py
def setup_clinical_embeddings():
    """Setup clinical embeddings with local model saving/loading"""
    
    # Check if model exists locally
    if LOCAL_MODEL_PATH.exists() and any(LOCAL_MODEL_PATH.iterdir()):
        try:
            clinical_emb = HuggingFaceEmbeddings(
                model_name=str(LOCAL_MODEL_PATH),
                encode_kwargs={"batch_size": 16}
            )
            # Test the model to ensure it's working
            test_vector = clinical_emb.embed_query("test medical query")
            print(f"Local model loaded successfully "
                  f"(test vector dim: {len(test_vector)})")
            return clinical_emb
        
        except Exception as e:
            print(f"Error loading local model: {e}")
            print("Downloading model...")
    
    # Download and save model locally
    LOCAL_MODEL_PATH.mkdir(parents=True, exist_ok=True)
    
    # Download using SentenceTransformer first
    model = SentenceTransformer(CLINICAL_MODEL_NAME)
    model.save(str(LOCAL_MODEL_PATH))
    print(f"Model saved to: {LOCAL_MODEL_PATH}")
    
    # LangChain embedding wrapper for SentenceTransformers
    clinical_emb = HuggingFaceEmbeddings(
        model_name=str(LOCAL_MODEL_PATH),
        encode_kwargs={"batch_size": 16}
    )
    
    return clinical_emb

def load_or_create_vectorstore():
    """Load existing vectorstore or create new one"""
    clinical_emb = setup_clinical_embeddings()
    data_provider = DataProvider()
    
    try:
        # Get chunked docs from appropriate source
        if data_provider.using_synthetic:
            chunked_docs = data_provider.load_chunked_docs()
        else:
            with open(CHUNKED_DOCS_PATH, "rb") as f:
                chunked_docs = pickle.load(f)
    except Exception as e:
        print(f"Error loading chunked documents: {e}")
        chunked_docs = None
    
    # Try to load existing vectorstore
    try:
        print("Loading existing vectorstore...")
        vectorstore = FAISS.load_local(
            VECTORSTORE_PATH,
            clinical_emb,
            allow_dangerous_deserialization=True
        )
        print("Vectorstore loaded successfully")
        return vectorstore, clinical_emb, chunked_docs
    
    except Exception as e:
        print(f"Error loading vectorstore: {e}")
        
        if chunked_docs is None:
            raise ValueError("No existing vectorstore found and "
                           "no chunked_docs provided to create new one")
        
        print("Creating new vectorstore...")
        vectorstore = FAISS.from_documents(chunked_docs, clinical_emb)
        vectorstore.save_local(VECTORSTORE_PATH)
        
        return vectorstore, clinical_emb, chunked_docs
\end{minted}

\subsection{Retrieval and Context Management}

The core RAG chatbot implementation with context-aware retrieval and anti-hallucination measures:

\begin{minted}{python}
# RAG_chat_pipeline/core/clinical_rag.py (key methods)
class ClinicalRAGBot:
    """Main RAG chatbot for clinical question answering"""
    
    def __init__(self, vectorstore, llm, embedder, chunked_docs=None):
        self.vectorstore = vectorstore
        self.llm = llm
        self.embedder = embedder
        self.chunked_docs = chunked_docs or []
        self.chat_history = []
        self.sources = []
        
        # Initialize retrieval chain
        self.retriever = self._create_contextual_retriever()
        self.chain = self._create_rag_chain()
    
    def _create_contextual_retriever(self):
        """Create context-aware retriever with filtering"""
        return self.vectorstore.as_retriever(
            search_type="mmr",  # Maximum marginal relevance
            search_kwargs={
                "k": DEFAULT_K,
                "fetch_k": DEFAULT_K * 2,
                "lambda_mult": 0.5  # Diversity parameter
            }
        )
    
    def ask_question(self, question: str, hadm_id: str = None, 
                    k: int = DEFAULT_K, search_strategy: str = "auto"):
        """
        Ask a question with optional admission-specific filtering
        
        Args:
            question: The clinical question to ask
            hadm_id: Optional admission ID for filtering
            k: Number of documents to retrieve
            search_strategy: 'auto', 'admission', or 'global'
        """
        start_time = time.time()
        
        try:
            # Entity extraction for auto-parameterization
            if ENABLE_ENTITY_EXTRACTION:
                extracted_entities = extract_entities(question, self.chat_history)
                if not hadm_id and extracted_entities.get("hadm_id"):
                    hadm_id = extracted_entities["hadm_id"]
            
            # Determine search strategy
            if search_strategy == "auto":
                search_strategy = "admission" if hadm_id else "global"
            
            # Retrieve relevant documents
            if search_strategy == "admission" and hadm_id:
                relevant_docs = self._admission_scoped_search(question, hadm_id, k)
            else:
                relevant_docs = self._global_search(question, k)
            
            if not relevant_docs:
                return self._no_documents_response(question)
            
            # Generate response using RAG chain
            response = self._generate_response(question, relevant_docs)
            
            # Post-processing and safety checks
            response = self._post_process_response(response, relevant_docs)
            
            search_time = time.time() - start_time
            
            return {
                "answer": response,
                "source_documents": relevant_docs,
                "search_time": search_time,
                "documents_found": len(relevant_docs),
                "search_strategy": search_strategy
            }
        
        except Exception as e:
            ClinicalLogger.error(f"Error in ask_question: {e}")
            return self._error_response(str(e))
    
    def _admission_scoped_search(self, query: str, hadm_id: str, k: int):
        """Search within specific admission context"""
        # Filter documents by admission ID
        admission_docs = [
            doc for doc in self.chunked_docs 
            if doc.metadata.get("hadm_id") == hadm_id
        ]
        
        if not admission_docs:
            ClinicalLogger.warning(f"No documents found for admission {hadm_id}")
            return []
        
        # Create temporary vectorstore for admission-specific search
        temp_vectorstore = FAISS.from_documents(admission_docs, self.embedder)
        temp_retriever = temp_vectorstore.as_retriever(search_kwargs={"k": k})
        
        return temp_retriever.get_relevant_documents(query)
    
    def _generate_response(self, question: str, documents: List[Document]):
        """Generate response using retrieved documents"""
        
        # Prepare context from retrieved documents
        context = self._format_context(documents)
        
        # Create clinical-focused prompt
        prompt = f"""
        Based on the following clinical information, answer the question accurately and concisely.
        Focus on factual information from the provided context.
        
        Clinical Context:
        {context}
        
        Question: {question}
        
        Instructions:
        - Provide specific clinical details when available
        - Include relevant values, dates, and medical codes
        - If information is incomplete, state what is missing
        - Always include the disclaimer about research/educational use
        """
        
        try:
            response = safe_llm_invoke(self.llm, prompt)
            return response
        except Exception as e:
            ClinicalLogger.error(f"LLM invocation failed: {e}")
            return "Unable to process the clinical query due to system error."
    
    def _post_process_response(self, response: str, documents: List[Document]):
        """Post-process response for safety and formatting"""
        
        # Ensure disclaimer is present
        if "MIMIC-IV" not in response:
            response += "\n\nData from MIMIC-IV database for research/education only."
        
        # Basic hallucination detection
        if self._has_potential_hallucination(response, documents):
            ClinicalLogger.warning("Potential hallucination detected in response")
            response = ("Based on the available clinical records, specific details "
                       "for this query may be limited. " + response)
        
        return response
    
    def _has_potential_hallucination(self, response: str, documents: List[Document]):
        """Basic hallucination detection using document grounding"""
        
        # Check if response contains medical codes not in source documents
        response_codes = re.findall(r'\b[A-Z]\d{2,5}\b', response)
        doc_content = " ".join([doc.page_content for doc in documents])
        
        for code in response_codes:
            if code not in doc_content:
                return True
        
        return False

    def chat(self, message: str, chat_history: list = None):
        """Handle conversational interactions with memory"""
        
        if chat_history:
            self.chat_history = chat_history
        
        # Extract context from chat history if available
        if ENABLE_ENTITY_EXTRACTION and self.chat_history:
            context = extract_context_from_chat_history(self.chat_history)
            enhanced_message = f"{message}\nContext: {context}" if context else message
        else:
            enhanced_message = message
        
        # Get response using standard ask_question method
        result = self.ask_question(enhanced_message)
        
        # Update chat history
        self.chat_history.extend([
            {"role": "user", "content": message},
            {"role": "assistant", "content": result["answer"]}
        ])
        
        # Trim chat history if too long
        if len(self.chat_history) > MAX_CHAT_HISTORY:
            self.chat_history = self.chat_history[-MAX_CHAT_HISTORY:]
        
        return result["answer"]
\end{minted}

\section{API Implementation}

\subsection{Flask Backend Endpoints}

The Flask API server provides RESTful endpoints for the frontend interface:

\begin{minted}{python}
# api/app.py
from flask import Flask, request, jsonify
from flask_cors import CORS
from RAG_chat_pipeline.core.main import main as initialize_clinical_rag

# Initialize Flask app
app = Flask(__name__, static_folder='../frontend/build')
CORS(app)  # Enable CORS for all routes

# Initialize RAG system
print("Initializing Clinical RAG System...")
try:
    chatbot = initialize_clinical_rag()
    print("Clinical RAG System initialized successfully")
except Exception as e:
    print(f"Error initializing Clinical RAG System: {e}")
    chatbot = None

@app.route('/api/chat', methods=['POST'])
def chat():
    """Handle chat requests"""
    if not chatbot:
        return jsonify({
            'error': 'RAG system not initialized'
        }), 500

    data = request.json
    if not data or 'message' not in data:
        return jsonify({
            'error': 'Missing message in request'
        }), 400

    user_message = data['message']
    chat_history = data.get('chat_history', [])

    # Debug logging
    print(f"API Chat Request:")
    print(f"  - Message: '{user_message}'")
    print(f"  - Chat history length: {len(chat_history) if chat_history else 0}")

    # Process with RAG system
    try:
        response = chatbot.chat(user_message, chat_history)
        
        print(f"API Chat Response:")
        print(f"  - Response length: {len(str(response))}")
        print(f"  - Response preview: {str(response)[:200]}...")

        return jsonify({
            'response': response,
            'sources': chatbot.sources if hasattr(chatbot, 'sources') else []
        })
    except Exception as e:
        print(f"Error processing message: {e}")
        return jsonify({
            'error': str(e)
        }), 500

@app.route('/api/question', methods=['POST'])
def ask_question():
    """Handle specific clinical questions"""
    if not chatbot:
        return jsonify({'error': 'RAG system not initialized'}), 500

    data = request.json
    if not data or 'question' not in data:
        return jsonify({'error': 'Missing question in request'}), 400

    question = data['question']
    hadm_id = data.get('hadm_id')
    k = data.get('k', 5)

    try:
        result = chatbot.ask_question(question, hadm_id=hadm_id, k=k)
        
        return jsonify({
            'answer': result['answer'],
            'source_documents': [
                {
                    'content': doc.page_content[:500],  # Truncate for API
                    'metadata': doc.metadata
                } for doc in result.get('source_documents', [])
            ],
            'search_time': result.get('search_time', 0),
            'documents_found': result.get('documents_found', 0)
        })
    except Exception as e:
        return jsonify({'error': str(e)}), 500

@app.route('/api/health', methods=['GET'])
def health_check():
    """Health check endpoint"""
    return jsonify({
        'status': 'healthy' if chatbot else 'unhealthy',
        'system': 'Clinical RAG API',
        'version': '1.0.0'
    })

if __name__ == '__main__':
    app.run(debug=True, host='0.0.0.0', port=5000)
\end{minted}

\subsection{Error Handling and Logging}

Comprehensive error management and logging system:

\begin{minted}{python}
# RAG_chat_pipeline/utils/logger.py
class ClinicalLogger:
    """Centralized logging utility with verbosity control"""
    LEVELS = {"quiet": 0, "error": 1, "warning": 2, "info": 3, "debug": 4}
    level = "info"

    @classmethod
    def set_level(cls, level: str):
        if level in cls.LEVELS:
            cls.level = level

    @classmethod
    def _enabled(cls, lvl: str) -> bool:
        return cls.LEVELS.get(cls.level, 3) >= cls.LEVELS.get(lvl, 3)

    @staticmethod
    def info(msg):
        if ClinicalLogger._enabled("info"):
            print(f"INFO: {msg}")

    @staticmethod
    def warning(msg):
        if ClinicalLogger._enabled("warning"):
            print(f"WARNING: {msg}")

    @staticmethod
    def error(msg):
        if ClinicalLogger._enabled("error"):
            print(f"ERROR: {msg}")

    @staticmethod
    def success(msg):
        if ClinicalLogger._enabled("info"):
            print(f"SUCCESS: {msg}")

    @staticmethod
    def debug(msg):
        if ClinicalLogger._enabled("debug"):
            print(f"DEBUG: {msg}")

# Error handling utility
class ErrorHandler:
    """Centralized error handling utility"""
    @staticmethod
    def safe_operation(func, *args, fallback=None, 
                      error_msg="Operation failed", **kwargs):
        try:
            return func(*args, **kwargs)
        except Exception as e:
            ClinicalLogger.warning(f"{error_msg}: {type(e).__name__}: {e}")
            return fallback

# Safe LLM invocation with retry logic
def safe_llm_invoke(llm, prompt, max_retries=3):
    """Safely invoke LLM with retry logic"""
    for attempt in range(max_retries):
        try:
            return llm.invoke(prompt)
        except Exception as e:
            ClinicalLogger.warning(f"LLM invocation attempt {attempt + 1} failed: {e}")
            if attempt == max_retries - 1:
                return "Unable to generate response due to system limitations."
            time.sleep(1)  # Brief pause before retry
\end{minted}

\section{Frontend Interface}

\subsection{React Chat Components}

The React frontend provides a Material-UI based chat interface:

\begin{minted}{javascript}
// frontend/src/hooks/useChat.js
import { useState, useCallback } from 'react';

export const useChat = () => {
  const [messages, setMessages] = useState([]);
  const [isLoading, setIsLoading] = useState(false);
  const [error, setError] = useState(null);

  const sendMessage = useCallback(async (message) => {
    if (!message.trim()) return;

    // Add user message to chat
    const userMessage = {
      id: Date.now(),
      text: message,
      sender: 'user',
      timestamp: new Date()
    };

    setMessages(prev => [...prev, userMessage]);
    setIsLoading(true);
    setError(null);

    try {
      const response = await fetch('/api/chat', {
        method: 'POST',
        headers: {
          'Content-Type': 'application/json',
        },
        body: JSON.stringify({
          message: message,
          chat_history: messages.map(msg => ({
            role: msg.sender === 'user' ? 'user' : 'assistant',
            content: msg.text
          }))
        }),
      });

      if (!response.ok) {
        throw new Error(`HTTP error! status: ${response.status}`);
      }

      const data = await response.json();

      if (data.error) {
        throw new Error(data.error);
      }

      // Add bot response to chat
      const botMessage = {
        id: Date.now() + 1,
        text: data.response,
        sender: 'bot',
        timestamp: new Date(),
        sources: data.sources || []
      };

      setMessages(prev => [...prev, botMessage]);
    } catch (err) {
      console.error('Chat error:', err);
      setError(err.message);
      
      // Add error message to chat
      const errorMessage = {
        id: Date.now() + 1,
        text: `Sorry, I encountered an error: ${err.message}`,
        sender: 'bot',
        timestamp: new Date(),
        isError: true
      };

      setMessages(prev => [...prev, errorMessage]);
    } finally {
      setIsLoading(false);
    }
  }, [messages]);

  const clearChat = useCallback(() => {
    setMessages([]);
    setError(null);
  }, []);

  return {
    messages,
    isLoading,
    error,
    sendMessage,
    clearChat
  };
};
\end{minted}

\subsection{Main Chat Component}

\begin{minted}{javascript}
// frontend/src/components/MessageList.js
import React from 'react';
import { 
  Box, 
  Paper, 
  Typography, 
  Chip, 
  Alert 
} from '@mui/material';
import { Message } from './Message';

export const MessageList = ({ messages, isLoading }) => {
  return (
    <Box
      sx={{
        flexGrow: 1,
        overflow: 'auto',
        p: 2,
        display: 'flex',
        flexDirection: 'column',
        gap: 2,
      }}
    >
      {messages.length === 0 && (
        <Alert severity="info" sx={{ mb: 2 }}>
          <Typography variant="body2">
            Welcome to the Clinical RAG Assistant! 
            Ask questions about patient data, lab results, diagnoses, 
            procedures, or medications.
          </Typography>
          <Box sx={{ mt: 1 }}>
            <Chip 
              label="Example: What lab values were abnormal for admission 25282710?" 
              variant="outlined" 
              size="small"
              sx={{ mr: 1, mb: 1 }}
            />
            <Chip 
              label="Example: Show me diagnostic information for admission 24711716" 
              variant="outlined" 
              size="small"
              sx={{ mr: 1, mb: 1 }}
            />
          </Box>
        </Alert>
      )}

      {messages.map((message) => (
        <Message key={message.id} message={message} />
      ))}

      {isLoading && (
        <Paper
          elevation={1}
          sx={{
            p: 2,
            alignSelf: 'flex-start',
            maxWidth: '70%',
            bgcolor: 'grey.100',
          }}
        >
          <Typography variant="body2" color="text.secondary">
            Processing your clinical query...
          </Typography>
        </Paper>
      )}
    </Box>
  );
};
\end{minted}

\section{Evaluation and Benchmarking System}

\subsection{Automated Model Evaluation}

The evaluation system for comparing model combinations:

\begin{minted}{python}
# RAG_chat_pipeline/benchmarks/rag_evaluator.py (key methods)
class ClinicalRAGEvaluator:
    """Configuration-driven RAG evaluator using config.py parameters"""

    def __init__(self, chatbot: ClinicalRAGBot):
        self.chatbot = chatbot
        self.patient_data = get_sample_data()
        self.results = []
        self.embedding_cache = {}

    def evaluate_question(self, gold_question: Dict, question_id: str = None):
        """Evaluate a single question with enhanced weighted scoring"""
        result = {
            "question": gold_question["question"],
            "category": gold_question["category"],
            "timestamp": datetime.now().isoformat(),
            "question_id": question_id
        }

        try:
            # Get chatbot response
            hadm_id = self._clean_hadm_id(gold_question.get("hadm_id"))
            enhanced_question = self._enhance_prompt(
                gold_question["question"], 
                gold_question["category"]
            )

            response = self.chatbot.ask_question(
                question=enhanced_question,
                hadm_id=hadm_id,
                k=5
            )

            # Multi-dimensional scoring
            factual_score = self.evaluate_factual_accuracy(gold_question, response)
            context_relevance = self.evaluate_context_relevance(
                gold_question["question"], 
                response.get("source_documents", [])
            )
            semantic_score = self.evaluate_semantic_similarity(
                response.get("answer", ""), 
                self._get_medical_keywords_for_category(gold_question["category"])
            )
            performance_score = self.evaluate_performance(response)

            # Weighted overall score
            overall_score = (
                factual_score * EVALUATION_SCORING_WEIGHTS["factual_accuracy"] +
                context_relevance * EVALUATION_SCORING_WEIGHTS["context_relevance"] +
                semantic_score * EVALUATION_SCORING_WEIGHTS["semantic_similarity"] +
                performance_score * EVALUATION_SCORING_WEIGHTS["performance"]
            )

            # Pass/fail determination
            category = gold_question["category"]
            threshold = EVALUATION_PASS_THRESHOLDS.get(category, 
                       EVALUATION_PASS_THRESHOLDS["default"])
            passed = overall_score >= threshold

            # Store detailed results
            result.update({
                "response": response.get("answer", ""),
                "overall_score": overall_score,
                "factual_accuracy_score": factual_score,
                "context_relevance_score": context_relevance,
                "semantic_similarity_score": semantic_score,
                "performance_score": performance_score,
                "pass_threshold": threshold,
                "passed": passed,
                "search_time": response.get("search_time", 0),
                "documents_found": response.get("documents_found", 0)
            })

            return result

        except Exception as e:
            result.update({
                "error": str(e),
                "overall_score": 0.0,
                "passed": False
            })
            return result

    def run_evaluation(self, questions: List[Dict]) -> Dict:
        """Run evaluation on a list of questions"""
        results = []
        total_time = 0

        for i, question in enumerate(questions):
            question_id = f"q_{i}"
            result = self.evaluate_question(question, question_id)
            results.append(result)
            total_time += result.get("search_time", 0)

        # Calculate summary statistics
        passed_count = sum(1 for r in results if r.get("passed", False))
        pass_rate = passed_count / len(results) if results else 0
        avg_score = sum(r.get("overall_score", 0) for r in results) / len(results)

        return {
            "summary": {
                "total_questions": len(questions),
                "passed": passed_count,
                "pass_rate": pass_rate,
                "average_score": avg_score,
                "total_time": total_time,
                "avg_time_per_question": total_time / len(results) if results else 0
            },
            "detailed_results": results,
            "category_breakdown": self._calculate_category_breakdown(results)
        }
\end{minted}

\subsection{Hallucination Detection and Safety Metrics}

\begin{minted}{python}
# RAG_chat_pipeline/results/enrich_results.py (safety metrics extraction)
def extract_efficiency_and_safety(payload: Dict[str, Any]) -> Tuple[float, float, float]:
    """Extract efficiency and safety metrics from evaluation results"""
    metrics = payload.get("metrics", {}) or {}
    notes = metrics.get("notes", "")
    
    # Extract total evaluation time
    m = re.search(r"completed in\s+([\d\.]+)s", notes)
    total_s = float(m.group(1)) if m else None

    # Get detailed question results
    detailed = (payload.get("raw_results", {}) or {}
                ).get("detailed_results", []) or []
    responses = [r.get("response", "") or "" for r in detailed 
                if r.get("response") is not None]

    # Calculate disclaimer rate
    disclaimer_hits = sum("Data from MIMIC-IV" in r for r in responses)
    disclaimer_rate = (disclaimer_hits / len(responses)) if responses else 0.0

    # Calculate hallucination rate based on factual accuracy threshold
    low_factual = [r.get("factual_accuracy_score", 0.0) for r in detailed]
    hallucination_rate = (sum(x < 0.6 for x in low_factual) / 
                         len(low_factual)) if low_factual else 0.0

    return total_s, disclaimer_rate, hallucination_rate
\end{minted}

\section{Configuration Management}

\subsection{Dynamic Model Configuration}

The configuration system allows dynamic switching between model combinations:

\begin{minted}{python}
# RAG_chat_pipeline/config/config.py (key configuration functions)
def set_models(embedding_model: str = "ms-marco", llm_model: str = "deepseek"):
    """
    Dynamically set the embedding and LLM models for the current session.
    
    Args:
        embedding_model: Embedding model nickname 
        llm_model: LLM model nickname
    
    Returns:
        tuple: (embedding_model, llm_model) that were set
    """
    global model_in_use, LLM_MODEL, CLINICAL_MODEL_NAME, LOCAL_MODEL_PATH, VECTORSTORE_PATH

    # Validate embedding model
    if embedding_model not in model_names:
        raise ValueError(f"Invalid embedding model: {embedding_model}. "
                        f"Available: {list(model_names.keys())}")

    # Validate LLM model
    if llm_model not in llms:
        raise ValueError(f"Invalid LLM model: {llm_model}. "
                        f"Available: {list(llms.keys())}")

    # Update global variables
    model_in_use = embedding_model
    LLM_MODEL = llms[llm_model]

    return embedding_model, llm_model

def get_config_summary():
    """Get a summary of current configuration."""
    llm_nickname = next(k for k, v in llms.items() if v == LLM_MODEL)
    return f"""
Current Configuration:
   Embedding Model: {model_in_use} ({CLINICAL_MODEL_NAME})
   LLM Model: {llm_nickname} ({LLM_MODEL})
   Vector Store: {VECTORSTORE_PATH.name}
   Model Path: {LOCAL_MODEL_PATH.name}
"""

# Model configurations for 54 combinations
model_names = {
    "ms-marco": ["S-PubMedBert-MS-MARCO", "pritamdeka/S-PubMedBert-MS-MARCO", 
                "faiss_mimic_sample1000_ms-marco"],
    "multi-qa": ["multi-qa-mpnet-base-cos-v1", "sentence-transformers/multi-qa-mpnet-base-cos-v1", 
                "faiss_mimic_sample1000_multi-qa"],
    "mini-lm": ["all-MiniLM-L6-v2", "sentence-transformers/all-MiniLM-L6-v2", 
               "faiss_mimic_sample1000_mini-lm"],
    "BioBERT": ["BioBERT-mnli-snli-scinli-scitail-mednli-stsb", 
               "pritamdeka/BioBERT-mnli-snli-scinli-scitail-mednli-stsb", 
               "faiss_mimic_sample1000_BioBERT"],
    # ... additional model configurations
}

llms = {
    "deepseek": "deepseek-r1:1.5b",
    "qwen": "qwen3:1.7b",
    "llama": "llama3.2:latest",
    "gemma": "gemma:2b",
    "phi3": "phi3:3.8b",
    "tinyllama": "tinyllama:1.1b",
}
\end{minted}

\section{Data Processing and Synthetic Data Generation}

\subsection{Synthetic Data Generation for Public Deployment}

\begin{minted}{python}
# synthetic_data/synthetic_data_generator.py (key methods)
class SyntheticDataGenerator:
    """Generate realistic synthetic clinical data matching MIMIC-IV structure"""
    
    def __init__(self, num_patients=100, num_admissions=150):
        self.num_patients = num_patients
        self.num_admissions = num_admissions
        self.fake = Faker()
        
    def generate_complete_dataset(self):
        """Generate complete synthetic dataset with all clinical tables"""
        
        # Generate base patient population
        patients = self._generate_patients()
        admissions = self._generate_admissions(patients)
        
        # Generate clinical data tables
        diagnoses = self._generate_diagnoses(admissions)
        procedures = self._generate_procedures(admissions)
        lab_events = self._generate_lab_events(admissions)
        prescriptions = self._generate_prescriptions(admissions)
        microbiology = self._generate_microbiology_events(admissions)
        
        return {
            'patients': patients,
            'admissions': admissions,
            'diagnoses': diagnoses,
            'procedures': procedures,
            'lab_events': lab_events,
            'prescriptions': prescriptions,
            'microbiology': microbiology
        }
    
    def _generate_realistic_lab_values(self, admission_type):
        """Generate realistic lab values based on admission context"""
        
        normal_ranges = {
            'Hemoglobin': (12.0, 16.0),
            'White Blood Cell Count': (4.5, 11.0),
            'Platelet Count': (150, 450),
            'Creatinine': (0.6, 1.2),
            'Glucose': (70, 110),
            'Sodium': (136, 145),
            'Potassium': (3.5, 5.0)
        }
        
        labs = []
        for lab_name, (min_val, max_val) in normal_ranges.items():
            # Introduce abnormalities based on admission type
            if admission_type == 'EMERGENCY':
                # Higher chance of abnormal values
                if random.random() < 0.4:
                    value = random.uniform(min_val * 0.7, min_val * 0.9) if random.random() < 0.5 else random.uniform(max_val * 1.1, max_val * 1.3)
                else:
                    value = random.uniform(min_val, max_val)
            else:
                # Mostly normal values
                value = random.uniform(min_val, max_val)
            
            labs.append({
                'itemid': hash(lab_name) % 10000,
                'label': lab_name,
                'value': round(value, 2),
                'valueuom': 'mg/dL' if 'Glucose' in lab_name or 'Creatinine' in lab_name else 'count',
                'flag': 'abnormal' if value < min_val or value > max_val else 'normal'
            })
        
        return labs
\end{minted}

This comprehensive code appendix provides implementation details for all major components of the Clinical RAG system, demonstrating the technical depth and architectural decisions that enable the system's clinical capabilities while maintaining code quality and safety standards.